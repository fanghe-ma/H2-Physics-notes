\documentclass[a4paper, 10pt]{article}
\usepackage[margin = 1in]{geometry}
\usepackage{amsmath}
\usepackage{tabularx}
\usepackage{framed}
\setlength{\parindent}{0em}

\begin{document}

\section*{Topic 4 - Forces}

\section{Hooke's Law}
A string or wire when stretched is found to obey Hooke's Law up to its limit of proportionality
\begin{framed}
   \textbf{Hooke's Law} states that the extension of a body is proportional to the applied load if the limit of proportionality is not exceeded
   \[
   F = kx
   \]
   
\end{framed}	

\[
   \text{Energy stored in the spring} = \text{work done in stretching the spring} = \text{Area unnder the force-extension graph}
\]
\[
   \text{Energy stored} = \frac{1}{2}Fx = \frac{1}{2}kx^2
\]

\section{Fluid Statics}
\subsection{Density}
\textbf{Density} of a substance is defined as its \textbf{mass per unit volume}

\[
\rho = \frac{m}{V}
\]
\subsection{Pressure}
\textbf{Pressure} is defined as the force per unit area, where the force is acting at right angles to the area \\

generally, 
\[
p = \frac{F}{A}
\]

Pressure in a fluid, or hydrostatic pressure is given by
\[
p = \rho gh
\]

total pressure at a given depth below the water is thus
\[
   P = p_{at} + \rho gh
\]

\subsection{Upthrust}
For a cylindrical object of cross sectional area $A$ submerged in a liquid of density $\rho$\\

The downward force on the top of the object due to fluid is 
\[
   F_{top} = p_1 A = \rho gh_1 A
\]
The upward force on the bottom of the object due to fluid is 
\[
   F_{bottom} = p_2 A = \rho g(h_1 + H)A
\]
Since upward force is greater than the downward force, there is thus a net force on the object
\begin{align*}
   F_{net} &= F_{bottom} - F_{top} \\
           &= \rho gAH \\
           &= \rho gV \\
           &= mg \text{, since $\rho = \frac{m}{V}$ where $m$ is the mass of \textbf{fluid} displaced}
\end{align*}	
\begin{framed}
   \textbf{The force of upthrust} is the vertical upward force exerted on a body by a fluid when it is fully or partially submerged in the fluid due to the difference in fluid pressure
\end{framed}	

\begin{framed}
   \textbf{Archimedes Principle} states that the buoyang force or upthrust is equal in magnitude and opposite in direction to the weight of the fluid that is displaced by a submerged or floating object
\end{framed}	

\section{Viscous force}
Viscous forces arise due to the collisions of an object which is moving through a fluid with the molecules of the fluid. 

Recall section on air resistance and net force on an object falling due to gravity from Topic 2 - Kinematics

\section{Equilibrium of forces}
For an object to be in equilibrium, it must be in both translational and rotational equilibrium

\begin{framed}
   \textbf{The conditions for equilibrium}
   \begin{enumerate}
      \item Resultant force on the object is zero
      \item Resultant moment on the object about any axis is zero
   \end{enumerate}	
\end{framed}

\subsection{1st Condition - Translational Equilibrium}
\[
\sum F_x = 0
\]
\[
\sum F_y = 0
\]

\subsection{2nd Condition - Rotational Equilibrium}
\begin{framed}
   \textbf{Principle of moments}: \\
   for any body in rotational equilibrium, the sum of all clockwise moment about any axis must be equal to the sum of all anticlockwise moment about the same axis
\end{framed}	

Turning effect of a force is called its moment
\begin{framed}
   the \textbf{moment} of a force about a point is defined as the produt of the force and the perpendicular distance from the point to the line of action of the force
\end{framed}	

Torque of a couple
\begin{framed}
   A couple consists of a pair of equal and opposite forces whose lines of action do not coincide \\ 
   The torque of a couple is the product of one force and the perpendicular distance between the two forces
   \[
   torque = Fd
   \]
   
\end{framed}	

\textbf{Note}: when only three coplanar forces act on a body in equilibrium, their lines of action must either all be parallel or they meet at a point

\section{Center of gravity}
\begin{framed}
   The \textbf{center of gravity} of a body is the point at which its weight or the resultant of the distributed gravitational attraction on the body appears to act.
\end{framed}	














\end{document}
