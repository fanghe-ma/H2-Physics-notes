\documentclass[a4paper, 10pt]{article}
\usepackage[margin = 1in]{geometry}
\usepackage{amsmath}
\setlength{\parindent}{0em}

\begin{document}

\section*{Topic 1 - Measurement}

\section{S.I. Base quantities and units}
Base quantities are fundamental physical quantities used to define other physical quantities. 

The Systeme Internationale d'Unites is based on seven base quantities  \\

\begin{center}
   \begin{tabular}{c|c|c}
      \hline 
      base quantity & base unit & Symbol \\
      \hline
      time & second & s \\
      legnth & metre & m \\
      mass & kilogram & kg \\
      current & ampere & A \\
      temperate & kelvin & K \\
      amount of substance & mole & mol \\
      luminous intensity & candela & cd \\
      \hline
   \end{tabular}	
\end{center}	

\section{Derived quantities and their units}
Derived quantities are physical quantities formed by combining base quantities. Derived units are products and/or quotients of base units.

Derived quantities are formed from base quantities according to a defining equation.

For example
\begin{center}
   \begin{tabular}{c|c|c|c|c}
      \hline
      Derived quantity & Defining equation & Base units & Derived unit & Symbol \\
      \hline
      force & force = mass $\times$ acceleration & $kgms^{-2}$ & newton & N \\
      \hline
   \end{tabular}	
\end{center}	

\section{Homogeneity of Equations}
Only quantities with the same base units can be added, subtracted or equated, i.e.
\[
   A (B+C) =  DE 
\]
\begin{center}
      $AB$, $AC$, and $DE$ have the same units
\end{center}	

When each of the terms in an equation has the same base units, the equation is \textbf{homogeneous} or \textbf{dimensionally consistent} \\

An equation can be dimensionally consistent but not physically correct due to
\begin{itemize}
   \item wrong coefficients / signs 
   \item missing / additional terms
\end{itemize}	

\section{Errors and Uncertaintines}
\subsection{Measuring instruments and their associated uncertainties}
   The precision of a instrument is determined by the number of \textbf{significant figures} in its measurements, which is in turn determined by \textbf{the smallest scale division}
   \begin{center}
      \begin{tabular}{c|c|c}
         \hline
         Physical Quantity & Instrument & Precision \\
         \hline 
         Length & metre rule & 0.1cm \\
                & vernier calipers & 0.01cm  \\
                & micrometer screw gauge & 0.001cm \\
         time & digital stopwatch & 0.01s \\
         mass & electronic balance & 0.01g \\
         \hline
      \end{tabular}
   \end{center}

\subsection{Systematic and random errors}
\textbf{Systematic errors} result in all readings being either always or above true value by a \textbf{fixed amount}
\begin{itemize}
   \item can be eliminated if source of error is known
   \item such as accounting for zero error
\end{itemize}	


\textbf{random errors} result in readings being scattered about true value, with erros having equal probability of being positive or negative

Random errors can be \textbf{reduced} by 
\begin{itemize}
   \item repeating the measurement and taking average value
   item plotting a graph and drawing line of best fit
\end{itemize}	

\subsection{Accuracy vs Precision}
\textbf{Accuracy}
\begin{itemize}
   \item degree of closeness of readings/mean reading to actual value
   \item affected by systematic error
\end{itemize}	


\textbf{Precision}
\begin{itemize}
   \item degree of agreement between repeated measurements
   \item affected by random error
\end{itemize}	

\begin{center}
   \begin{tabular}{c|c c}
      \hline
      & Accuracy & Precision \\
      \hline
      Instrument & Calibration of instrument & smallest scale division \\ \\
      Measurements & Closeness of mean to true value & closeness of measurements to one another \\
      \hline
   \end{tabular}
\end{center}

\section{Derived uncertainties}
For 2 independent measurements of $X$ and $Y$, $X_1$ $Y_1$, each with uncertainty $\Delta X$ and $\Delta Y$  let variable $Z$ be $X \times Y$, find the associated uncertainty $\Delta Z$ 
\subsection{the upper-lower bound method of uncertainty propagation}
\begin{align*}
   \Delta Z &= \frac{Z_{max} - Z_{min}}{2} \\
            &= \frac{X_{max} \times Y_{max} - X_{min} \times Y_{min}}{2}
\end{align*}	

\subsection{calcualtion of uncertainties of derived quantities}
\begin{center}
   \begin{tabular}{c|c}
      $Q = aX \pm bY$ & $\Delta Q = |a|\Delta X + |b| \Delta Y$ \\ \\
      $Q = aX \times Y$ & $\frac{\Delta Q}{Q} = \frac{\Delta X}{X} + \frac{\Delta Y}{Y} $  \\ \\
      $Q = a\frac{X}{Y}$ & $\frac{\Delta Q}{Q} = \frac{\Delta X}{X} + \frac{\Delta Y}{Y} $  \\ \\
      $Q = aX^m \times bY^n$ & $\frac{\Delta Q}{Q} = |m|\frac{\Delta X}{X} + |n|\frac{\Delta Y}{Y} $  \\ \\
      $Q = \frac{aX^m}{bY^n}$ & $\frac{\Delta Q}{Q} = |m|\frac{\Delta X}{X} + |n|\frac{\Delta Y}{Y} $
   \end{tabular}
\end{center}

\textbf{absolute uncertainty} $\Delta Q$ are expressed to \textbf{1sf} \\

\textbf{fractional uncertainty} $\frac{\Delta Q}{Q}$ are expressed to \textbf{2sf} \\

percentage uncertainty of x is 
\[
   \frac{\Delta Q}{Q} \times 100\%
\]










\end{document}	
