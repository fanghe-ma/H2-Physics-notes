\documentclass[a4paper, 10pt]{article}
\usepackage[margin = 1in]{geometry}
\usepackage{amsmath}
\usepackage{tabularx}
\usepackage{framed}
\setlength{\parindent}{0em}
\usepackage{import}
\usepackage{pdfpages}
\usepackage{transparent}
\usepackage{xcolor}

\newcommand{\incfig}[2][1]{%
    \def\svgwidth{#1\columnwidth}
    \import{./figures/}{#2.pdf_tex}
}

\pdfsuppresswarningpagegroup=1

\begin{document}

\section*{Topic 3 - Dynamics}
\section{Newton's first law of motion}

\begin{framed}
   \textbf{\textit{Newton's First law of motion}}: a body continues in its state of rest or uniform motion in a straight line unless acted upon by a resultant external force
\end{framed}	

\subsection{Inertia}
The inertia of a body is its reluctance to a change in motion  \\

The mass $m$ of a body is an intrinsic property which resists change in motion

\subsection{Equilirbium}
the conditions necessary for equilibrium of a body are 
\begin{itemize}
   \item resultant force acting on body is 0
   \item resultant torque on the body about any axis is 0
\end{itemize}	


\section{Newton's second law of motion and Momentum}
\begin{framed}
   \textbf{\textit{Newton's second law of motion}}: rate of change of momentum of a body is proportional to the resultant force acting on it and occurs in the direction of the force
  \[
     F_{net} \propto \frac{d(p)}{dt}
  \]
   
\end{framed}	

\subsection{Momentum}
Momentum of a body is defined as the product of its mass $m$ and velocity $v$, ie 
\[
   p = mv 
\]

\subsection{Newton's second law}
\begin{align*}
   F_{net} &\propto \frac{d(p)}{dt} \\
   F_{net} &= k\frac{d(p)}{dt} \text{, and since k = 1 when quantities in S.I. units, } \\
   F_{net} &=  \frac{d(p)}{dt} = \frac{d(mv)}{dt} \\
           &= m\frac{dv}{dt} + v\frac{dm}{dt} 
\end{align*}	

Hence 
\begin{itemize}
   \item when mass is constant, $F_{net} = m\frac{dv}{dt}$ 
   \item when velocity is constamt, $F_{net} = v\frac{dm}{dt}$ 
\end{itemize}

\section{Newton's third law of motion}
\begin{framed}
   \textbf{\textit{Newton's Third law of motion}}: if body A exerts a force on body B, then body B exerts an equal but opposite force on A 
   \[
      |F_{by\  A\  on\  B}| = |F_{by\  B\  on\  A}|
   \]
   \[
      F_{AB} = -F_{BA}
   \]
\end{framed}	

\section{Impulse and momentum change}
Impulse is equal to the change in momentum, defined as the product of a force $F$ acting on an object and the time $\Delta t$ for which the force acts

\[
   F = \frac{\Delta p}{\Delta t}
\]
\[
\Delta p = F \Delta t
\]

Impulse can be found as the area under the $F-t$ graph
\section{Collisions and Conservation of momentum}
By Newton's Third Law, when $A$ collides with $B$, 
\[
F_{BA} = -F_{AB}  
\]
since time interval $dt$ is the same, impulse on $A$ is opposite to impulse on $B$ 
\[
   \int_{t_i}^{t_f} F_{BA} dt = - \int_{t_i}^{t_f} F_{AB} dt
\]

since impulse = change in momentum
\[
  \Delta p_a = \Delta p_b
\]
\[
   m_A v_A + m_A u_A = -(m_B v_B - m_B u_B)
\]
\[
   m_A u_A + m_B u_B = m_A v_A + m_B v_B
\]

\begin{center}
   net initial momentum of bodies = net final momentum of bodies
\end{center}	

\begin{framed}
  \textbf{The principle of conservation of linear momentum}: when bodies in a system interact, total momentum of the system remains constant, provided no net external force acts on the system   

  \[
  \sum m_i u_i = \sum m_i v_i
  \]
  
\end{framed}	

Types of collision
\begin{itemize}
   \item \textbf{Elastic collision} total KE conserved
   \item \textbf{Inelastic collision} total KE not conserved
   \item \textbf{Completely inelastic collision} particles have the same final velocity (particles stick together and move off together)
\end{itemize}	

for elastic collision, by principle of conservation of linear momentum
\[
m_1 u_1 + m_2 u_2 = m_1 v_1 + m_1 v_2
\]

and since that total KE remains constant
\[
   \frac{1}{2} m_1 u_1^2 + \frac{1}{2} m_2 u_2^2 =
   \frac{1}{2} m_1 v_1^2 + \frac{1}{2} m_2 v_2^2
\]
\[
   m_1 u_1^2 + m_2 u_2^2 =
   m_1 v_1^2 + m_2 v_2^2
\]

rearranging the expressions
\begin{align*}
   m_1 (u_1 - v_1) = m_2 (v_2 - u_2) \\
   m_1 (u_1^2 - v_1^2) = m_2(v_2^2 - u^2_2)
\end{align*}	
dividing the equations to remove $m_1$ and $m_2$ 
\[
 \frac{u_1^2 - v_1^2}{u_1 - v_1} = \frac{u_2^2- v2^2}{v_2 - u_2}
\]
\[
   \frac{(u_1+v_1)(u_1 -v_1)}{u_1 - v_1} = \frac{(v_2+u_2)(v_2 - u_2)}{v_2 - u_2}
\]
\[
u_1 + v_1 = v_2 + u_2
\]
\[
u_1 - u_2 = v_2 - v_1
\]
\begin{center}
   relative speed of approach = relative speed of separation
\end{center}	






\end{document}	
