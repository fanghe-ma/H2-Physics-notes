\documentclass[a4paper, 10pt]{article}
\usepackage[margin = 1in]{geometry}
\usepackage{amsmath}
\usepackage{tabularx}
\usepackage{framed}
\setlength{\parindent}{0em}
\newcolumntype{L}{>{\arraybackslash}m{7cm}}

\begin{document}

\section*{Topic 7 - Gravitational field} 
\section{Gravitational force}
\begin{framed}
   \textbf{Newton's law of gravitation} states that two point masses attract each other with a force that is directly proportional to the product of their masses and inversely proportional to the square of the distance between them
\[
F = \frac{GMm}{r^2}
\]
where G is the gravitational constant with a value of $6.67 \times 10^{-11} Nm^2 kg^{-2}$ 
\end{framed}	

\section{Gravitational field strength}
\subsection{Gravitational field}
\begin{framed}
   \textbf{A gravitational field} is a region of space in which a mass experiences a gravitational force
\end{framed}	

\subsection{Gravitational field strength}
\begin{framed}
   The \textbf{gravitational field strength} at a point in space is defined as the gravitational force experienced per unit mass at that point
   \[
   g = \frac{F}{m}
   \]

   \[
   g = \frac{GM}{r^2}
   \]

   note that the resultant gravitational field strength at a point due to more than one mass is the \textbf{vector sum} of the individual gravitational field strengths due to each mass
\end{framed}	

\subsection{Gravitational field strength of a uniform sphere}
According to the \textbf{Shell Theorem}
\begin{enumerate}
   \item a spherically symmetric body affects external objects gravitationally as though all of its mass were concentrated at a point in its center (behaves like \textbf{point mass} 
   \item If the body is a spherically symmetric shell, no net gravitational force is exerted by the shell on any objects inside the shell 
   \item for a solid sphere of constant density, gravitational force \textbf{varies linearly} with distance from the center
\end{enumerate}	


henec for a uniform solid sphere of radius R, the variations of $g$ with distance from the center, $r$ is as follows

\begin{itemize}
   \item Inside the sphere, for $r \le R$: 
      \begin{itemize}
         \item mass of the inner sphere of radius $r$ is $M = \rho V = \rho \frac{4}{3}\pi r^3$
         \item $g$ due to inne sphere is thus
            \[
           g = \frac{GM}{r^2} = \frac{G}{r^2} \frac{4}{3}\rho \pi r^3 = \frac{4}{3} G \rho\pi r
            \]
            \[
            g \propto r
            \]
            
            
      \end{itemize}	
   \item Outside the sphere, for $r \ge R$: 
      \[
      g = \frac{GM}{r^2}
      \]
\end{itemize}	
\subsection{Gravitational field strength of earth and apparent weight}
For an object of mass $m$ held by a force $T$ provided by a spring balance \\

\begin{center}
   \begin{tabular}{L|L}
      At polar region & At equator \\
      \hline
      $F_R = F_g - T$  &
      $F_R = F_g - T$  \\ 
                       & \\
      There is no circular motionm, $F_R$ = 0 & To provide for centripetal force, $F_R = F_c$ \\
      & \\
      $T = F_g$ & $ T = F_g - F_c$ \\
      spring balance indicates true weight of object & spring balance indicates apparent weight which is smaller than true weight of object \\
      \[
         mg_{freefall} = mg
      \] 
      & 
      \[
         mg_{freefall} = mg - ma_c
      \]
      
   \end{tabular}
\end{center}


\section{Gravitational Potential Energy}
\begin{framed}
   the \textbf{gravitational potential energy} of a mass at a point in a gravitational field is the work done by an external force in bringing the mass from infinity to that point
   \[
   U = - \frac{GMm}{r}
   \]
\end{framed}	
\begin{itemize}
   \item GPE at infinity is defined as 0
   \item work done by external force in bringing an object from infinity to r is 

      \[
         W = \int_{\infty}^r F_{ext} dr = \int_{\infty}^{r} \frac{GMm}{r^2} = - \frac{GMm}{r}
      \]
   \item $U$ and $\phi$ are negative because gravitational force is attractive, hence to bring a mass from infinity to a point a field, the direction of the external force is opposite to the direction of displacement of mass, henec negative work is done by the external force
\end{itemize}	

\subsection{Relationship between force and potential energy}
\[
F = - \frac{dU}{dr}
\]

\section{Gravitational Potential}

\begin{framed}
   the \textbf{gravitational potential} at a point in a gravitational field is defined as the work done per unit mass by an external force in bringing a small test mass from infinity to that point
   \[
   \phi = \frac{U}{m} = - \frac{GM}{r}
   \]
   
\end{framed}	

\subsection{relationship between field strength and potential}
\[
g = - \frac{d\phi}{dr}  
\]


\section{Escape Velocity}
\textbf{Escape velocty} is the minimum speed needed for the object to escape the gravitational influence of earth \\

At infinity, $E_p$ = 0, if an object has sufficient energy to just reach infinity, $E_k = 0$, hence $E_T E_p + E_k = 0$ \\

By principle of conservation of energy, an object with total energy of zero will be able to just reach infinity \\

For an object of mass $m$ with initial velocity $v$, its initial energy is given by
\[
   E_p = U = - \frac{GMm}{r_{earth}}
\]
\[
E_k = \frac{1}{2} mv^2
\]

Energy required to reach infinity is such that
\[
 E_T \ge 0
\]
\[
E_p + E_k \ge 0
\]
\[
    - \frac{GMm}{r_{earth}}  + \frac{1}{2}mv^2 \ge 0
\]
Hence escape velocity is 
\[
   v \ge \sqrt{ \frac{2GM}{R_{earth}}}
\]

\[
   v \ge \sqrt{2gR_{earth}}
\]



\section{Circular Orbits}
recall that for an object in circular motion with mass $m$ and linear velocity $v$ and angular velocity $\omega$ 
\[
F_c = \frac{mv^2}{r} = mr\omega^2
\]

and that the time for one complete revolution is the period $T$, given by
\[
T = \frac{2\pi}{\omega} = \frac{2\pi r}{v}
\]


\subsection{Planetary motion and Kepler's third law}

for a planet in orbit of the Sun, gravitational force $F_g$ provides for centripetal force $F_c$ hence 
\begin{align*}
   F_g &= F_c \\ 
   \frac{GMm}{r^2} &= mr\omega^2 \\ 
   \frac{GMm}{r^2} &= mr \left( \frac{2\pi}{T}\right)^2 \\ 
   T^2 &= \frac{4\pi^2r^3}{GM} \\ \\
   T^2 &\propto r^3
\end{align*}	


\subsection{Satellite motions}
\textbf{Energy of a satellite} \\
Since gravitational force provides for centripetal force
\begin{align*}
   F_g &= F_c \\
   \frac{GMm}{r^2} &= m \frac{v^2}{r} \\
   v^2 = \frac{GM}{r}
\end{align*}	

hence KE is given by
\[
E_k = \frac{1}{2} mv^2 = \frac{1}{2} \frac{GMm}{r} 
\]

GPE is given by
\[
E_p = - \frac{GMm}{r}
\]

Total energy is thus 
\[
E_p + E_k = -\frac{GMm}{2r}
\]

\textbf{Geostationary orbits} is one that remains at a fixed position in the sky as viewed from any location on earth's surface, satistying the following conditions 
\begin{enumerate}
   \item Its orbital period is the same as that of Earth about its axis of rotation (24 hours)
   \item Its direction of rotation is the same as that of Earth (west to east)
   \item its plane of orbit lies in the same plane as the equator 
\end{enumerate}	

Based on these conditions, using 
\[
   \frac{GMm}{r^2} = mr \left( \frac{2\pi}{T}\right)^2 
\]

\[
   r = \frac{T^2 GM}{4\pi^2}^{\frac{1}{3}} = 42250km
\]

\begin{center}
   \begin{tabular}{L|L}
      Advantages & Disadvantages \\
      \hline
      \begin{enumerate}
         \item continuous surveillance of the region underneath 
         \item easy for communicating with ground station as it is permanently in view. no adjustment of ground absed antenna necessary
         \item due to high altitude, satellites can transmit and receive signals over a large range
      \end{enumerate}	
       & distance from earth surface is large, leading to 
      \begin{enumerate}
         \item significant loss of signal strengths
         \item poorer resolution in imaging satellites
         \item time lag in telecommunication 
      \end{enumerate}	\\
   \end{tabular}
\end{center}

\subsection{Binary star systems}
For two stars of mass $M$ and $m$, separated by a distance $d$, show that their period of circular motion is \[
   T^2 = \frac{4\pi^2 d^3}{G(M+m)}
\]

both stars revolve around some center of mass $C$, such that their respective radius of circular motion is $r_1$ and $r_2$, and they have the same angular velocity $\omega$  \\

the gravitational force between them is given by 
\[
F_g = \frac{GMm}{d^2}
\]

hence 
\[
\frac{GMm}{d^2} = m \omega r_1^2,\ and\  \frac{GMm}{d^2} = M \omega r_2^2
\]
\[
mr_1 = Mr_2
\]
\[
r_2 = \frac{m}{M}r_2
\]

\[
   d = r_1 + r_2 = r_1 + \frac{m}{M}r_1 = r_1 \left( 1 + \frac{m}{M} \right)
\]

substituting $r_1 = \frac{d}{1 + \frac{m}{M}}$ 
\[
   \frac{GMm}{d^2} = m \left( \frac{2 \pi}{T}\right)^2 r_1
\]

\[
   T^2 = \frac{4\pi^2 d^2}{GM}\frac{d}{1 + \frac{m}{M}} = \frac{4\pi^2 d^3}{G(M+m)}
\]

 
 









\end{document}	
