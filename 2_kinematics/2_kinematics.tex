\documentclass[a4paper, 10pt]{article}
\usepackage[margin = 1in]{geometry}
\usepackage{amsmath}
\usepackage{tabularx}
\setlength{\parindent}{0em}

\begin{document}

\section*{Topic 2 - Kinematics}

\section{Basic quantities}

\begin{center}
   \begin{tabularx}{\textwidth}{X|X}
      \hline
      Scalar & Vector \\
      \hline
      \hline
      \textbf{\textit{Distance, x}} & \textbf{\textit{Displacement, s}} \\
      the total length of path followed by object & The distance moved in a specified direction from a reference point \\
      \hline
      \textbf{\textit{speed, v}} & \textbf{\textit{velocity, v}} \\
      instantaneous speed is the rate of change of distance wrt time
      & instantaneous velocity is the rate of change of displacement wrt time \\
      $v = \frac{dx}{dt}$
      &
      $v = \frac{ds}{dt}$
       \\
      \hline
      average speed is the total distance travelled over total time taken &
      average velocity is the total change in displacement over total time taken \\
      \[
         \langle v \rangle = \frac{\Delta x}{\Delta t}
      \] &
      \[
         \langle v \rangle = \frac{\Delta x}{\Delta t}
      \] \\
      \hline
      & \textbf{\textit{Acceleration, a}} \\
      & instantaneous acceleration is the rate of change of velocity wrt time \\
      & \[
         a = \frac{dv}{dt}
      \] \\ 
      \hline
      & Average acceleration is the total change in velocity over total time \\
      & \[
         \langle a \rangle = \frac{\Delta v}{\Delta t}
      \]  \\
   \end{tabularx}
\end{center}

\section{Equations for uniformly accelerated motion}
\begin{eqnarray}
   v = u + at \\
   s = \frac{1}{2} (u+v) t \\
   s = ut + \frac{1}{2} at^2 \\
   v^2 = u^2 + 2as
\end{eqnarray}	

\section{kinematics of free fall and the effect of air resistance}
objects in the uniform gravitational field of earth undergo uniformly accelerated motion downwards, and experience a constant acceleration with magnitude
\[
   g = 9.81ms^{-2}
\]


Objects experience air resistance, whose magnitude is proportional to velocity and whose direction is opposite to velocity
\begin{itemize}
   \item on an object's way up, it experiences air resistance in the direction of downward acceleration due to gravity, hence \[
   a_{up} > g
   \]
   \item on its way down, it experiences air resistance opposite gravity, hence \[
         a_{down} < g
   \]
   
\end{itemize}	


\section{non-linear motion}


\end{document}	
